\documentclass[12pt]{beamer}
\usetheme{Warsaw}
\usepackage[utf8]{inputenc}
\usepackage[english]{babel}
\usepackage{amsmath}
\usepackage{stmaryrd}
\usepackage{amsfonts}
\usepackage{amssymb}
\usepackage{graphicx}
\usepackage{xcolor}
\usepackage[backend=bibtex, style=authoryear, maxcitenames=2]{biblatex}
\addbibresource{../../Thesis/references.bib} 
\renewcommand*{\bibfont}{\footnotesize}
\usepackage{url}
\definecolor{lila}{RGB}{128,0,128}

\newcommand{\myCite}[1]{{\scriptsize\parencite{#1}}}

\setbeamertemplate{navigation symbols}{}
\setbeamertemplate{footline}{%
	\raisebox{5pt}{%
		\makebox[\paperwidth]{%
			\hfill\makebox[10pt]{%
				\textcolor{gray}{\footnotesize\insertframenumber}
			}
		}
	}
}

\definecolor{regal}{RGB}{81,0,0}
\usepackage{array}
\usepackage{colortbl}
\makeatletter
\newcommand{\thickhline}{%
    \noalign {\ifnum 0=`}\fi \color{regal}\hrule height 4pt
    \futurelet \reserved@a \@xhline
}
\newcolumntype{"}{@{\color{regal}\hskip\tabcolsep\vrule width 8pt\hskip\tabcolsep}}
\makeatother

% listings
\usepackage[TS1,T1]{fontenc}
\usepackage{newunicodechar}
\newcommand*\longs{{\fontencoding{TS1}\selectfont s}}
\newunicodechar{ſ}{\longs}
\usepackage{listings}
\lstset{
	captionpos=b,
	frame=single,
	breaklines=true,
	tabsize=2,
	aboveskip=4pt,
	belowskip=6pt,
	literate=%
		{Ö}{{\"O}}1
		{Ä}{{\"A}}1
		{Ü}{{\"U}}1
		{ß}{{\ss}}1
		{ü}{{\"u}}1
		{ä}{{\"a}}1
		{ö}{{\"o}}1
		{~}{{\textasciitilde}}1,
	alsoletter={{\"u}}
}
\definecolor{maroon}{rgb}{0.5,0,0}
\definecolor{darkgreen}{rgb}{0,0.5,0}
\lstdefinelanguage{XML}
{
	basicstyle=\ttfamily\tiny,
	morestring=[s]{"}{"},
	morecomment=[s]{?}{?},
	morecomment=[s]{!--}{--},
	commentstyle=\color{darkgreen},
	moredelim=[s][\color{black}]{>}{<},
	moredelim=[s][\color{red}]{\ }{=},
	stringstyle=\color{blue},
	identifierstyle=\color{maroon}
}

%\setbeamercovered{transparent} 
%\logo{}  
%\subject{} 
\begin{document} 
		
\begin{frame}[plain,noframenumbering]
	\begin{minipage}{0.1\textwidth}
		\hspace{-0.5cm}\includegraphics[scale=0.049]{Images/Buchruecken}
	\end{minipage}
	\begin{minipage}{0.6\textwidth}
		\begin{center}
			\vspace{-1.5cm}
			\Large{Extracting recipe ingredients \\ from cookbooks} \\
			\vspace{0.5cm}
			\large{Irrlichter} \\
			\small{von \\ \vspace{-0.1cm} Torsten Knauf}
		\end{center}
	\end{minipage}
	\begin{minipage}{0.25\textwidth}
		\arrayrulecolor{red}
		\small
		\begin{tabular}{"l}
			Der Beginn einer \\ Master-Arbeit \\
			\thickhline
			\color{white} \\
			Irrlichter \\
			\thickhline 
			\color{white} - \\
			\color{white} - \\
			\thickhline 
			\color{white} - \\
			\color{white} - \\
			\thickhline 
			\color{white} - \\
			\color{white} - \\
			\thickhline 
			\color{white} - \\
			\color{white} - \\
			\thickhline 
			\color{white} - \\
			\color{white} - 
		\end{tabular}
	\end{minipage}
\end{frame}

\begin{frame}
	\tableofcontents
\end{frame}

\begin{frame}[fragile]{Beispiel cueML}
	\begin{lstlisting}[language=XML]
	<cue:recipe type="Suppen." rcp-id="B-16">
	<head>Mock Turtle Suppe.</head>
	
	<p>Es wird hierzu für <cue:yield atLeast="24" atMost="30">24—30 Personen</cue:yield>	eine kräftige <ref target="#Bouillon">Bouillon</ref> von 8—10 Pfund	<cue:recipeIngredient ref="#Rindkochfleisch" atLeast="8" atMost="10"	unit="Pfund">Rindfleisch</cue:recipeIngredient> mit <cue:recipeIngredient ref="#Wurzelwerk">Wurzelwerk</cue:recipeIngredient> gekocht.  [...]</p>
	
	<note>Anmerk. Der <cue:recipeIngredient ref="#Englische_Soja" optional="True" 
	>Soja</cue:recipeIngredient> macht die Suppe gewürzreicher, kann jedoch gut
	wegbleiben, und statt <cue:recipeIngredient ref="#Madeira" altGrp="1"
	>Madeira</cue:recipeIngredient> kann man <cue:recipeIngredient
	ref="weißen_Franzwein" altGrp="2">weißen Franzwein</cue:recipeIngredient>
	und etwas <cue:recipeIngredient ref="#Rum" altGrp="2" quantity="etwas"
	>Rum</cue:recipeIngredient> nehmen<cue:alt target="1 2"/>. Sowohl die Bouillon als Kalbskopf können schon am	vorhergehenden Tage, ohne Nachtheil der Suppe, gekocht werden.</note>
	</cue:recipe>
	\end{lstlisting}
\end{frame}

\begin{frame}[fragile]{Wie man es hätte auch machen können I}
	\begin{lstlisting}[language=XML]
	und statt <cue:recipeIngredient ref="#Madeira" altGrp="1">Madeira</cue:recipeIngredient> kann man <cue:recipeIngredient ref="weißen_Franzwein" altGrp="2">weißen Franzwein</cue:recipeIngredient> und etwas <cue:recipeIngredient ref="#Rum" altGrp="2" quantity="etwas">Rum</cue:recipeIngredient> nehmen <cue:alt target="1 2"/>.
	
		VS
	
	und statt <recipeIngredient ref="#Madeira" xml:id="B-16-Madeira">Madeira</recipeIngredient> kann man <recipeIngredient ref="weißen_Franzwein" xml:id="B-16-Franzwein">weißen Franzwein</recipeIngredient> und etwas <recipeIngredient ref="#Rum" xml:id="B-16-Rum">Rum</recipeIngredient> nehmen. Sowohl die Bouillon als Kalbskopf können schon am vorhergehenden Tage, ohne Nachtheil der Suppe, gekocht werden.
	
	<recipeIngredientGrp xml:id="B-16-alt1" target="#B-16-Madeira"/>
	<recipeIngredient xml:id="B-16-alt2" target="#B-16-Franzwein #B-16-Rum"/>
	<alt target="#B-16-alt1 #B-16-alt2" mode="excl"/>
	\end{lstlisting}
\end{frame}

\begin{frame}[fragile]{Wie man es hätte auch machen können II}
	\begin{lstlisting}[language=XML]
	8—10 Pfund	<cue:recipeIngredient ref="#Rindkochfleisch" atLeast="8" atMost="10"	unit="Pfund">Rindfleisch</cue:recipeIngredient>
	
	VS
	
	<atLeast target="B-16-Rindfleisch">8</atLeast>—<atMost target="1">10</atMost> <unit target="B-16-Rindfleisch">Pfund</unit> <ingredient xml:id="B-16-Rindfleisch" ref="#Suppenrindfleisch">Rindfleisch</ingredient>
	\end{lstlisting}
\end{frame}

\begin{frame}[fragile]{Trainingsdaten}
	\vspace{-0.7em}\begin{lstlisting}
für			...	O
24			... B-Yield
—			... I-Yield
30			... I-Yield
Personen	...	O
kräftige	... O
Bouillon	... O
von			...	O
8			...	B-Quantity
—			...	I-Quantity
10			...	I-Quantity
Pfund		...	B-Unit
Rindfleisch	... B-Ingredient
mit			...	O
Wurzelwerk	...	B-Ingredient
gekocht		...	O
.			...	O
	\end{lstlisting}
\end{frame}	

\begin{frame}{Ergebnis vom Prototyp}
	wort | wahrheit | getaggt
\end{frame}


\begin{frame}{Nächste Schritte}
	\begin{itemize}
		\item Mehr Rezepte labeln
		\item Label-Engineering
		\item Feature-Engineering
		\item Skript um Zutatenliste zu extrahieren
	\end{itemize}
\end{frame}
	
\section{}
\begin{frame}[t,allowframebreaks]{Literaturverzeichnis}
	\printbibliography[heading=none]
\end{frame}

\end{document}